\documentclass[a4paper,11pt,toc]{report}
\usepackage{tikz}
\usetikzlibrary{calc,positioning}
\usepackage{pgfplots}
\usepackage{csquotes}
\usepackage{caption}
\usepackage{array}
\usepackage{booktabs}

\title{Basic Physics for electrical engineering}
\author{Hendrik Theede}
\date{\today}


\usepackage[linktoc=all]{hyperref}
\hypersetup{linkcolor=green}

\begin{document}
\maketitle
\begin{abstract}
    This file contains a small summary of the contents of the \texttt{Physik für Elektrotechnik} course at the University of Rostock.
    Might include unnecessary commentary and will not be peer reviewed. 
%
    Written in \TeX{} with reference to Prof.\ Hage (whose course i attended in 2021/22).
\end{abstract}

\tableofcontents
\listoffigures
\listoftables

\setcounter{chapter}{-1}
\chapter[Basics]{Units and Equations}
\subsection*{Units and Unit-Prefixes}
There exists a small set of SI-Units, which together make up all other known physical units. 
SI itself refers to the french \textit{Système international d'unites}, which simply means \textbf{International System for units}.
A unit itself is a way to measure a specific amount of a physically described perceived part of our life (dimensions like time, lengths/space, temperature, mass, amounts, \ldots).
Table~\ref{tab:siunits} provides an overview.

\begin{table}
    \centering
    \begin{tabular}{l l l l}
        \toprule[3pt]
            Dimension & SI-unit (symbol) & unit & unit (symbol)\\
        \midrule[2pt]
            time & $t$ & second & $s$ \\
            length & $l$ & metre & $m$ \\
            mass & $m$ & gramm & $g$ \\
            current (electrical) & $I$ & Ampere & $A$ \\
            temperature (thermodynamic) & $T$ & Kelvin & $K$ \\
            amount of substance & $n$ & mol & $mol$ \\
            light's strength & $I_v$ & second & $s$ \\
        \bottomrule[3pt]
    \end{tabular}
    \caption{List of the known SI-Units.}\label{tab:siunits}
\end{table}


\end{document}